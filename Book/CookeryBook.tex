\documentclass[a4paper,12pt]{report}

\usepackage{graphicx, subfigure,anysize,epsfig}
\usepackage{fullpage}
\usepackage{pdfpages}
\usepackage{listofsymbols}
\usepackage{multirow}
\usepackage[pdftex]{hyperref}
\usepackage[english]{babel}
\usepackage{tikz}
%\usepackage{mathdesign}
\usepackage{wrapfig}
\usepackage{color}
\usepackage{enumerate}
%\usepackage{enumitem}
\usepackage{paralist}
\usepackage{framed}
\usepackage{fancybox} %kaders

\usepackage{amsmath,amssymb,mathrsfs}
\usepackage{amsthm}
\usepackage{wasysym}
\setlength{\parindent}{0pt}
\usepackage{blindtext}


\title{Recipes for and from PhD students}
\date{2018\\ }
\author{Gabriela Diaz and Roel Tielen}

\begin{document}
\maketitle
\tableofcontents

\chapter{Thai noodles with chicken}

\section*{Ingredients}
\begin{figure}[h]

\begin{minipage}{0.65\textwidth}
\begin{itemize}
\item Chicken ($250$ gr.)
\item Thai vegetables ($400$ gr.)
\item Noodles ($125$ gr.)
\item Conimex `Woksaus' Sweet and Sour
\end{itemize}
\end{minipage}
\begin{minipage}{0.3\textwidth}
	\includegraphics[width=35mm,scale=0.07]{Images/noodles_ingredients.jpg}
\end{minipage}
\end{figure}


\section*{Method}

\begin{figure}[h]

\begin{minipage}{0.5\textwidth}
Bake the chicken in a baking pan for a couple of minutes untill the chicken is nice light brown. In the meantime, put the noodles in boiling water for a couple of minutes in a seperate pan. \\
\\
Add the vegetables to the chicken and let it bake for another $5$-$10$ minutes (depending on your own preference). Drain the noodles and them together with the sauce. \\
\\
Mix everyhting and... that's it!

\end{minipage}
\begin{minipage}{0.4\textwidth}
	\includegraphics[scale=0.065]{Images/noodles.jpg}
\end{minipage}
\end{figure}

\newpage

\chapter{Ravioli with mushrooms and lentils}

\section*{Ingredients}
\begin{figure}[h]

\begin{minipage}{0.65\textwidth}
\begin{itemize}
\item Ravioli ($250$ gr.)
\item Hak bolognese `schotel' ($500$ gr.)
\item Mushrooms ($400$ gr.)
\item Parmezan cheese ($50$ gr.)
\end{itemize}
\end{minipage}
\begin{minipage}{0.3\textwidth}
	\includegraphics[scale=0.17]{Images/ravioliend.jpg}
\end{minipage}
\end{figure}


\section*{Method}

\begin{figure}[h]

\begin{minipage}{0.6\textwidth}
Cut the mushrooms and bake them in a baking pan for five minutes untill they are nice brown. In the meantime, put the ravioli in boiling water for approximately two minutes in a seperate pan.  \\
\\
Add the bolgenese `schotel' and let it bake for another five minutes on a medium heat. Add the ravioli and mix everything. Finish the dish with some nice cheese on top!

\end{minipage}
\begin{minipage}{0.35\textwidth}
	\includegraphics[scale=0.065]{Images/ravioli.jpg}
	\includegraphics[scale=0.065]{Images/mushroom.jpg}
\end{minipage}
\end{figure}


\chapter{Tortelinni with Italian vegetables}

\section*{Ingredients}
\begin{figure}[h]

\begin{minipage}{0.65\textwidth}
\begin{itemize}
\item Tortelinni ($250$ gr.)
\item Meat (`gehakt') ($300$ gr.)
\item Italian vegetables mix ($400$ gr.)
\item Bolognese sauce
\end{itemize}
\end{minipage}
\begin{minipage}{0.3\textwidth}
	\includegraphics[scale=0.07]{Images/ing.jpg}
\end{minipage}
\end{figure}


\section*{Method}

\begin{figure}[h]

\begin{minipage}{0.6\textwidth}
Bake the meat in a frying pan for five minutes untill its nice lightbrown. In the meantime, put the tortelinnii in boiling water for approximately two minutes in a seperate pan. Look on the packing for the precise time :-)  \\
\\
Add the vegetables and bake it for 7 minutes. Then add the bolognese sauce and let it bake for another five minutes on a medium heat. Add the tortellini and mix everything. Finish the dish with some nice cheese on top!

\end{minipage}
\begin{minipage}{0.35\textwidth}
	\includegraphics[scale=0.065]{Images/tor.jpg}
	\includegraphics[scale=0.065]{Images/end.jpg}
\end{minipage}
\end{figure}


\chapter{Vegetarian Lasagna}


\section*{Ingredients}
\begin{figure}[h]

\begin{minipage}{0.65\textwidth}
\begin{itemize}
\item Lasagna leaves
\item Vegetarian meat 
\item Mushrooms, courgette, onions
\item Ricotta cheese and cheese snippets
\item Tomato sauce
\end{itemize}
\end{minipage}
\begin{minipage}{0.3\textwidth}
	\includegraphics[scale=0.07]{Images/las.jpg}
\end{minipage}
\end{figure}


\section*{Method}

\begin{figure}[h]

\begin{minipage}{0.6\textwidth}
Bake the vegetarian meat in a frying pan and add the tomato sauce after some minutes. In the meantime, cut the mushrooms, onions and courgette and add it to the meat. Mix the ricotta cheese with $2/3$ of the cheese snips.  Let the sauce boil for about ten minutes.\\ \\
Then, start with a thin layer of sauce and on top lasagna leaves for the bottom. Afterwards, make layers consisting of sauce, then cheese and finally leaves. Push the top leaves a little bit downwards so that they are covered in sauce. Finish the top with the remaining cheese snips. Put the lasagna in the oven for about $55$ minutes. 
\end{minipage}
\begin{minipage}{0.35\textwidth}
	\includegraphics[scale=0.065]{Images/lasag.jpg}
	\includegraphics[scale=0.065]{Images/lasagna.jpg}
\end{minipage}
\end{figure}


\chapter{Vegetarian Couscous}


\section*{Ingredients}
\begin{figure}[h]

\begin{minipage}{0.65\textwidth}
\begin{itemize}
\item Couscous ($2$ gr.)
\item Cucumber, Cherry tomatoes 
\item Paprika, Onion ($2$ pieces each)
\item Conimex `Sweat and Sour
\item Feta cheese ($200$ gr.)
\item Raisins, Cashewnuts
\end{itemize}
\end{minipage}
\begin{minipage}{0.3\textwidth}
	\includegraphics[scale=0.07]{Images/cous.jpg}
\end{minipage}
\end{figure}


\section*{Method}

\begin{figure}[h]

\begin{minipage}{0.6\textwidth}
Cut the onions, bake them in a frying pan and add the sauce after some minutes. In the meantime, add some boiled water to the couscous (not to much!), and let it rest for some minutes. Cut all the vegetables (tomatoes, cucumber, paprika) and add them to the onions and the sauce. \\
\\
Then, add the couscous and mix everything. Cut the cheese, and add it together with the raisins and the cashewnuts. Thats it! 
\end{minipage}
\begin{minipage}{0.35\textwidth}
	\includegraphics[scale=0.065]{Images/couscous.jpg}
\end{minipage}
\end{figure}


\end{document}