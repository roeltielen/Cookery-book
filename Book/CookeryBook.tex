\documentclass[a4paper,12pt]{report}

\usepackage{graphicx, subfigure,anysize,epsfig}
\usepackage{fullpage}
\usepackage{pdfpages}
\usepackage{listofsymbols}
\usepackage{multirow}
\usepackage[pdftex]{hyperref}
\usepackage[english]{babel}
\usepackage{tikz}
%\usepackage{mathdesign}
\usepackage{wrapfig}
\usepackage{color}
\usepackage{enumerate}
%\usepackage{enumitem}
\usepackage{paralist}
\usepackage{framed}
\usepackage{fancybox} %kaders

\usepackage{amsmath,amssymb,mathrsfs}
\usepackage{amsthm}
\usepackage{wasysym}
\setlength{\parindent}{0pt}
\usepackage{blindtext}


\title{Recipes for and from PhD students}
\date{2018\\ }
\author{Gabriela Diaz and Roel Tielen}

\begin{document}
\maketitle
\tableofcontents

\chapter{Thai noodles with chicken}

\section*{Ingredients}
\begin{figure}[h]

\begin{minipage}{0.59\textwidth}
\begin{itemize}
\item Chicken ($250$ gr.)
\item Thai vegetables ($400$ gr.)
\item Noodles ($125$ gr.)
\item Conimex `Woksaus' Sweet and Sour
\end{itemize}
\end{minipage}
\begin{minipage}{0.4\textwidth}

	\includegraphics[scale=0.09]{Images/noodles_ingredients.jpg}
\end{minipage}
\end{figure}


\section*{Method}
Bake the chicken in a baking pan for a couple of minutes untill the chicken is nice light brown. In the meantime, put the noodles in boiling water for a couple of minutes in a seperate pan. 
Add the vegetables to the chicken and let it bake for another $5$-$10$ minutes (depending on your own preference). Drain the noodles and them together with the sauce. Mix everyhting and... that's it!
\end{document}